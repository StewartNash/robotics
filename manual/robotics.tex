\documentclass[11pt]{book}

\usepackage{amsmath}
\usepackage{amssymb}
\usepackage{amsthm}
\usepackage{mathtools}

\newtheoremstyle{example} % Style name
	{10pt} % Space above
	{10pt} % Space below
	{\normalfont} % Body font
	{} % Header font
	{\bfseries} % Header font
	{.} % Separator
	{10pt} % Space after header
	{} % Header text
\theoremstyle{example}
\newtheorem{example}{Example}[section]
\newtheorem{definition}{Definition}[section]

\begin{document}

\title{Phoenix Dynamics Robotics Manual}
\author{Stewart Nash}

\maketitle

\tableofcontents

\mainmatter

\chapter{Dynamics}

\section{Kinematics}

Take a vector with components $\mathbf{P}=(a_x,b_y,c_z)$. A scale factor $w$ can be added (to the matrix form) to give
\begin{equation}
	\mathbf{P}=
	\begin{bmatrix}
		P_x\\
		P_y\\
		P_z\\
		w
	\end{bmatrix}
\end{equation}
where $(a_x,b_y,c_z)=(P_x/w,P_y/w,P_z/w)$. A direction vector can be represented by a scale factor of zero ($w=0$).

A universe reference frame is represented by $F_{x,y,z}$ and a moving frame is represented by $F_{n,o,a}$ where the letters n, o, and a come from the words normal, orientation and approach. Relative to the gripper, the $z$-axis is the approach axis by which the gripper approaches an object. The orientation with which the gripper frame approaches the part is the orientation axis. The normal-axis or x-axis is normal to both. A fourth vector which gives the location of a frame relative to a reference frame can be added to the vectors representing the components of the $n$-, $o$-, and $a$-axes to give a homogeneous matrix representation of this relative frame
\begin{equation}
	F=
	\begin{bmatrix}
		1&0&0&d_x\\
		0&1&0&d_y\\
		0&0&1&d_z\\
		0&0&0&1
	\end{bmatrix}
\end{equation}
Pre-multiplying the frame matrix by the transformation matrix will yield the new location of the frame.

Rotation matrices about the $x$-, $y$- and $z$-axes are given by
\begin{equation}
	\begin{aligned}
		\mathrm{rot}(x,\theta)&=
		\begin{bmatrix}
			1&0&0\\
			0&\cos{\theta}&\sin{\theta}\\
			0&\sin{\theta}&\cos{\theta}
		\end{bmatrix}\\
		\mathrm{rot}(y,\theta)&=
		\begin{bmatrix}
			\cos{\theta}&0&\sin{\theta}\\
			0&1&0\\
			-\sin{\theta}&0&\cos{\theta}
		\end{bmatrix}\\
		\mathrm{rot}(z,\theta)&=
		\begin{bmatrix}
			\cos{\theta}&-\sin{\theta}&0\\
			\sin{\theta}&\cos{\theta}&0\\
			0&0&1
		\end{bmatrix}
	\end{aligned}
\end{equation}
Denoting the transformation of frame $R$ relative to frame $U$ (universe) as $\prescript{U}{}T_R$, denoting the $p$ relative to the frame $R$ as $\prescript{R}{}p=p_{noa}$, and denoting $p$ relative to frame $U$ as $\prescript{U}{}p=p_{xyz}$ we have
\begin{equation}
	\prescript{U}{}p=\prescript{U}{}T_R\times\prescript{R}{}p
\end{equation}

\subsection{Twists and Screws}

The special orthogonal group is denoted $SO$ and we may define it as the space of rotation matrices in $\mathbb{R}^{n{\times}n}$ by
\begin{equation}
	SO(n)=\{R\in\mathbb{R}^{n{\times}n}:RR^T=I,\mathrm{det}\,R=+1\}
\end{equation}
The space of $n{\times}n$ skew-symmetric matrices is given by
\begin{equation}
	so(n)=\{S\in\mathbb{R}^{n{\times}n}:S^T=-S\}
\end{equation}
The special Euclidean group, generalized to $n$ dimensions, is given by
\begin{equation}
	SE(n)\equiv\mathbb{R}^n\times{SO(n)}
\end{equation}
In other words, the special Euclidean group is comprised of a configuration pair $(p, R)$ which is the product space of $\mathbb{R}^n$ with $SO(n)$ and is given in 3 dimensions by
\begin{equation}
	SE(3)=\{(p,R):p\in\mathbb{R}^3,R\in{SO(3)}\}=\mathbb{R}^3\times{SO(3)}
\end{equation}
We can define $se(n)$ as being comprised of the configuration pair $(v, \omega)$ where $v$ is an element of $\mathbb{R}^n$ and $\omega$ is a skew-symmetric matrix from $so(n)$. In three dimensions we have
\begin{equation}
	se(3)\equiv\{(v,\hat{\omega}):v\in\mathbb{R}^3,\hat{\omega}\in{so(3)}\}
\end{equation}
A \emph{twist} is an element of $se(3)$, $\hat{\xi}\in{se(3)}$. The twist coordinates of $\hat{\xi}$ are given by $\xi\equiv(v,\omega)$.

A rigid body motion which consists of rotation about an axis in space through an angle of $\theta$ radians, followed by a translation along the same axis by and amount $d$ is referred to as a screw motion. A \emph{screw} is composed of an axis $l$, a pitch $h$, and a magnitude $M$.

A generalized force acting on a rigid body consists of a linear component, i.e. a `pure force', and an angular component, i.e. a `pure moment', acting at a point. This generalized force can  be represented as a vector in $\mathbb{R}^6$
\begin{equation}
	F=
	\begin{bmatrix}
		f\\
		\tau
	\end{bmatrix}
\end{equation}
where $f$ is the linear component in $\mathbb{R}^3$ and $\tau$ is the rotational component in $\mathbb{R}^3$. We refer to this force and moment pair as a \emph{wrench}.

\section{Differential Motion}

\section{Dynamic Analysis}

The Lagrangian is given by $L=T-V$ where $T$ and $V$ are the kinetic and potential energy of a system, respectively. If $F_i$ is the summation of all external forces acting on the $i$th generalized coordinate $q_i$, the equations of motion are given by
\begin{equation}
	F_i=\frac{d}{dt}\left(\frac{\partial{L}}{\partial\dot{q}_i}\right)-\frac{\partial{L}}{\partial{q_i}}
\end{equation}
When this is solved, the resulting equations of manipulator dynamics can be written
\begin{equation}
	M(q)\ddot{q}+V(q,\dot{q})+G(q)=\tau
\end{equation}
or, alternatively,
\begin{equation}
	M(q)\ddot{q}+C(q,\dot{q})\dot{q}+N(q,\dot{q})=\tau
\end{equation}
where $M(q)$ is the inertia matrix, $V(q,\dot{q})$ is the Coriolis and centripetal acceleration vector, $C(q,\dot{q})$ is the Coriolis matrix, $G(q)$ is the gravity vector, $N(q,\dot{q})$ represents gravity and other non-linear terms and $\tau$ is the $n$-vector of generalized forces which could be, for example, the actuator torques.

To be concrete, we can give an example of this in which we have two coordinates, $x$ and $\theta$, for the $i$th of $n$ linkages. If $F_i$ is the summation of all external forces for a linear motion and $T_i$ is the summation of all external torques for a rotational motion, then
\begin{equation}
	\begin{aligned}
		F_i&=\frac{d}{dt}\left(\frac{\partial{L}}{\partial\dot{x}_i}\right)-\frac{\partial{L}}{\partial{x_i}}\\
		T_i&=\frac{d}{dt}\left(\frac{\partial{L}}{\partial\dot{\theta}_i}\right)-\frac{\partial{L}}{\partial\theta_i}
	\end{aligned}
\end{equation}
We can simplify the equations of motion for a 2-DOF system which is given by
\begin{equation}
	\begin{bmatrix}
		\tau_i\\
		\tau_j
	\end{bmatrix}=
	\begin{bmatrix}
		M_{ii}&M_{ij}\\
		M_{ji}&M_{jj}
	\end{bmatrix}
	\begin{bmatrix}
		\ddot{\theta}_i\\
		\ddot{\theta}_j
	\end{bmatrix}+
	\begin{bmatrix}
		C_{iii}&C_{ijj}\\
		C_{jii}&C_{jjj}
	\end{bmatrix}
	\begin{bmatrix}
		\dot{\theta}_i\\
		\dot{\theta}_j
	\end{bmatrix}+
	\begin{bmatrix}
		C_{iij}&C_{iji}\\
		C_{jij}&C_{jji}
	\end{bmatrix}
	\begin{bmatrix}
		\dot{\theta}_i\dot{\theta}_j\\
		\dot{\theta}_j\dot{\theta}_i
	\end{bmatrix}+
	\begin{bmatrix}
		G_i\\
		G_j
	\end{bmatrix}
\end{equation}
Where the coefficient $M_{ii}$ is the effective inertia at joint $i$, such that an acceleration at joint $i$ causes a torque at joint $i$ equal to $M_{ii}\ddot{\theta}_i$, and the coefficient $M_{ij}$ is the coupling inertia between joints $i$ and $j$ such that an acceleration at joint $i$ or $j$ causes a torque at joint $j$ or $i$ equal to $M_{ij}\ddot{\theta}_j$ or $M_{ji}\ddot{\theta}_i$. $C_{ijj}\dot{\theta}_j^2$ terms represent centripetal forces acting at joint $i$ due to a velocity at joint $j$. All terms with $\dot{\theta}_i\dot{\theta}_j$ represent Coriolis accelerations and, when multiplied by corresponding inertias, represent Coriolis forces. $G_i$ represents gravity forces at joint $i$.

\section{Trajectory Planning}

\chapter{Controls}

\section{State Variable Representation}

Powerful tools from matrix algebra can be used to solve sets of first-order differential equations. That is why it is sometimes helpful to transform a system described by \emph{n}th-order differential equations into a system of first-order differential equations.

\subsection{Systems Modeled by Linear Differential Equations}

Consider a system represented by a \emph{n}th-order, single-input linear constant coefficient differential equation
\begin{equation}
	\sum_{i=0}^n{a_i\frac{d^iy}{dt^i}}=u
\end{equation}
This equation can be replaced by \emph{n} first-order differential equations
\begin{equation}
	\left\{
		\begin{aligned}
			\frac{dx_k}{dt}&=x_{k+1},\,1<k<n\\
			\frac{dx_n}{dt}&=\frac{1}{a_n}\left[\sum_{k=0}^{n-1}{a_kx_{k+1}}\right]+\frac{1}{a_n}u
		\end{aligned}
	\right.
\end{equation}
where $x_1\equiv{y}$ and $i,k\in\mathbb{W}$. This can be written as a matrix equation
\begin{equation}
	\begin{bmatrix}
		\frac{dx_1}{dt}\\
		\frac{dx_2}{dt}\\
		\vdots\\
		\frac{dx_n}{dt}
	\end{bmatrix}
	\begin{bmatrix}
		0&1&0&\cdots&0\\
		0&0&1&\cdots&0\\
		\vdots&\vdots&\vdots&\ddots&\vdots\\
		-\frac{a_0}{a_n}&-\frac{a_1}{a_n}&-\frac{a_2}{a_n}&\cdots&-\frac{a_{n-1}}{a_n}
	\end{bmatrix}
	\begin{bmatrix}
		x_1\\
		x_2\\
		\vdots\\
		x_n
	\end{bmatrix}+
	\begin{bmatrix}
		0\\
		0\\
		\vdots\\
		0\\
		\frac{1}{a_n}
	\end{bmatrix}u
\end{equation}
or
\begin{equation}
	\frac{d\mathbf{x}}{dt}=A\mathbf{x}+\mathbf{b}u
\end{equation}
A multi-input-multi-output (MIMO) system can be represented by
\begin{equation}
	\begin{bmatrix}
		\frac{dx_1}{dt}\\
		\frac{dx_2}{dt}\\
		\vdots\\
		\frac{dx_n}{dt}
	\end{bmatrix}
	\begin{bmatrix}
		a_{11}&a_{12}&\cdots&a_{1n}\\
		a_{21}&a_{22}&\cdots&a_{2n}\\
		\vdots&\vdots&\ddots&\vdots\\
		a_{n1}&a_{n2}&\cdots&a_{nn}
	\end{bmatrix}
	\begin{bmatrix}
		x_1\\
		x_2\\
		\vdots\\
		x_n
	\end{bmatrix}+
	\begin{bmatrix}
		b_{11}&b_{12}&\cdots&b_{1r}\\
		b_{21}&b_{22}&\cdots&b_{2r}\\
		\vdots&\vdots&\ddots&\vdots\\
		b_{n1}&b_{n2}&\cdots&b_{nr}
	\end{bmatrix}
	\begin{bmatrix}
		u_1\\
		u_2\\
		\vdots\\
		u_r
	\end{bmatrix}
\end{equation}
or
\begin{equation}
	\frac{d\mathbf{x}}{dt}=A\mathbf{x}+B\mathbf{u}\label{eq:3_25}
\end{equation}
where $\mathbf{u}$ is an \emph{r}-vector of input functions.\\
Let $\mathbf{\Phi}$ be the $n\times{n}$ \emph{transition matrix} of the differential equation given above which is described by the matrix equation
\begin{equation}
	\frac{d\mathbf{\Phi}}{dt}=A\mathbf{\Phi}
\end{equation}
If $\mathbf{\Phi}(0)=I$ (initial condition) then $\mathbf{\Phi}(t)=e^{At}$ where
\begin{equation}
	e^{At}=\sum_{n=0}^\infty{\frac{A^nt^n}{n!}}
\end{equation}
The solution to \eqref{eq:3_25} on the interval $0\leq{t}<\infty$ is given by
\begin{equation}
	\mathbf{x}(t)=e^{At}\mathbf{x}(0)+\int_0^t{e^{A(t-\tau)}B\mathbf{u}(\tau)\,d\tau}
\end{equation}

\subsection{Systems Modeled by Constant Coefficient Linear Difference Equations}

An $n$-th order (linear constant-coefficient) difference equation is given by
\begin{equation}
	\sum_{i=0}^n{a_iy(k+i)}=\sum_{i=0}^m{b_iu(k+i)}
\end{equation}
Define a shift operator by the equation
\begin{equation}
	Z[y(k)]\equiv{y(k+1)}
\end{equation}
The $n$-th order linear constant-coefficient difference equation
\begin{equation}
	y(k+n)+\sum_{i=0}^{n-1}{a_iy(k+i)}=u(k)
\end{equation}
can be written as
\begin{equation}
	(Z^n+\sum_{i=0}^{n-1}{a_iZ^i})[y(k)]=u(k)
\end{equation}
The characteristic equation of this difference equation is
\begin{equation}
	Z^n+\sum_{i=0}^{n-1}{a_iZ^i}=0
\end{equation}

\section{Signal Flow Graphs}

\section{Stability, Sensitivity and Error}

\subsection{Stability}

\begin{definition} A continuous system \emph{stable} if its impulse response $y_\delta(t)$ approaches zero as time approaches infinity. Similarly, a discrete-time system is stable if its Kronecker delta response $y_\delta(k)$ approaches zero as time approaches infinity.\end{definition}

A continuous or discrete-time system can also be defined as stable if every bounded input results in a bounded output.

\subsection{Sensitivity}

Sensitivity can be given for either the transfer or the frequency response function. The sensitivity of a system to its parameters is a measure of how much either of these system functions differ from its nominal when each of its parameters differs from its nominal value. Sensitivity can also be given for systems expressed in the time domain.

For a mathematical model $T(k)$ with $k$ regarded as the only parameter, the sensitivity of $T(k)$ with respect to the parameter $k$ is defined by
\begin{equation}
	S_k^{T(k)}\equiv\frac{d\,\ln{T(k)}}{d\,\ln{k}}=\frac{dT(k)}{dk}\frac{k}{T(k)}
\end{equation}

\subsection{Error}

For the canonical feedback system, the open-loop transfer function is given by
\begin{equation}
	GH=\frac{Ks^a\prod_{i=1}^{m-a}{(s+z_i)}}{s^b\prod_{i=1}^{n-b}{s+p_i}}
\end{equation}
We only consider the case where $b\geq{a}$ and $l\equiv{b-a}$.

A canonical system whose open-loop transfer function can be written in the form
\begin{equation}
	GH=\frac{K\prod_{i=1}^{m-a}{(s+z_i)}}{s^l\prod_{i=1}^{n-a-l}{s+p_i}}\equiv\frac{KB_1(s)}{s^lB_2(s)}
\end{equation}
where $l\geq{0}$ and $-z_i$ and $-p_i$ are the nonzero finite zeros and poles of $GH$, respectively, is called a type $l$ system.

Three criteria of the effectiveness (of feedback) in a stable type $l$ unity feedback system are
\begin{itemize}
	\item position (step) error constant
	\item velocity (ramp) error constant
	\item acceleration (parabolic) error constant
\end{itemize}

\subsection{Specifications}

We define an open-loop frequency response function $GH(\omega)$. For continuous systems $GH(\omega)\equiv{GH(j\omega)}$ and for discrete-time systems $GH(\omega)\equiv{GH(e^{j\omega{T}})}$. There are seven frequency-domain specifications which we will cover:
\begin{itemize}
	\item Phase crossover frequency, $\omega_\pi$
	\item Gain margin
	\item Gain crossover frequency, $\omega_\mathrm{1}$
	\item Phase margin, $\phi_\mathrm{PM}$
	\item Delay time, $T_d$
	\item Cutoff frequency, $\omega_c$ or $f_c$
	\item Bandwidth, BW
	\item Cutoff rate
	\item Resonance peak, $M_p$
	\item Resonant frequency, $\omega_p$
\end{itemize}

When using time-domain specifications, we define them in terms of responses to either unit step, ramp or parabolic inputs. We look at both steady state and transient responses. Steady state performance specifications include $K_p$, $K_v$ and $K_a$. The transient response performance specifications which we will cover are as follows:
\begin{itemize}
	\item Overshoot
	\item Delay time, $T_d$
	\item Rise time, $T_r$
	\item Settling time, $T_s$
	\item Dominant time constant, $\tau$
\end{itemize}

\section{Nyquist Analysis and Design}

\subsection{Mapping}

Let us consider a complex variable $s=\sigma+j\omega$. We will denote a complex transfer function of $s$ as $P(s)$. Let us also consider a complex variable $z=\mu+j\nu$ and denote a discrete-time (system) complex transfer function of $z$  as $P(z)$. For the first variable and transfer function we create two graphs: (1) the $s$-plane which has $j\omega$ on the ordinate and $\sigma$ on the abscissa, (2) the $P(s)$-plane which has $\mathrm{Im}\,P$ on the ordinate and $\mathrm{Re}\,P$ on the abscissa. The function $P$ maps points of the $s$-plane into the $P(s)$-plane. Similarly, $P(z)$ is a mapping or transformation from the $z$-plane to the $P(z)$-plane. For Nyquist stability plots, the locus of points in the $s$-plane which are chosen to map is called the Nyquist path. A polar plot is constructed in the $P(s)$-plane by taking $s=0+j\omega$.  

\subsection{Examples}

\begin{example}
	Consider a system with the open-loop transfer function
	\begin{equation}
		GH_1(s)=\frac{K}{s(s+p_1)(s+p_2)}\quad{K_1,p_1,p_2>0}
	\end{equation}

	We create the Nyquist (polar) plot using the following code
\begin{verbatim}
import numpy
import matplotlib.pyplot as plt
import control

K1 = 1
p1 = 0.5
p2 = 1

numerator = K1
denominator = numpy.poly([0, -p1, -p2])
GH = control.TransferFunction(numerator, denominator)

plt.figure()
control.nyquist(GH, omega_limits=(0.01, 100), omega_num=1000)
plt.show()

omega = numpy.logspace(-2, 2, 100)
magnitude, phase, w = control.frequency_response(GH, omega)

plt.figure()
ax = plt.subplot(111, projection='polar')
ax.plot(phase[0], magnitude[0])
ax.set_theta_zero_location("N")
ax.set_theta_direction(-1)
ax.grid(True)
plt.show()
\end{verbatim}

\end{example}

\begin{example}
	The general transfer function of a continuous sytem lead compensator is
	\begin{equation}
		P_\mathrm{Lead}(s)=\frac{s+a}{s+b}\quad{b>a}
	\end{equation}
	This compensator has a zero at $s=-a$ and a pole at $s=-b$. The general transfer function of a continuous system lag compensator is
	\begin{equation}
		P_\mathrm{Lag}(s)=\frac{a(s+b)}{b(s+a)}\quad{b>a}
	\end{equation}
	However, in this case the zero is at $s=-b$ and the pole is at $s=-a$. The gain factor $a/b$ is included because of the way it is usually mechanized. The general transfer function of a continuous system lag-lead compensator is
	\begin{equation}
		P_\mathrm{LL}(s)=\frac{(s+a_1)(s+b_2)}{(s+b_1)(s+a_2)}\quad{b_1>a_1,b_2>a_2}
	\end{equation}
	This compensator has two zeros and two poles. For mechanization considerations, the restriction $a_1b_2=b_1a_2$ is usually imposed.
	
	Say that we have a continuous system with an open-loop frequency response function given by
	\begin{equation}
		GH(s)=\frac{K_1}{s(s+p_1)(s+p_2)}\quad{p_1,p_2,K_1>0}
	\end{equation}
	
\end{example}

\begin{example}
	The general transfer function of a digital lead compensator is
	\begin{equation}
		P_\mathrm{Lead}(z)=\frac{K_\mathrm{Lead}(z-z_c)}{z-p_c}\quad{z_c>p_c}
	\end{equation}
	This compensator has a zero at $z=z_c$ and a pole at $z=p_c$. Its steady state gain is
	\begin{equation}
		P_\mathrm{Lead}(1)=\frac{K_\mathrm{Lead}(1-z_c)}{1-p_c}
	\end{equation}
	The gain factor $K_\mathrm{Lead}$ is included in the transfer function to adjust its gain at a given $\omega$ to a desired value. The general transfer fucntion of a digital lag compensator is
	\begin{equation}
		P_\mathrm{Lag}(z)=\frac{(1-p_c)(z-z_c)}{(1-z_c)(z-p_c)}\quad{z_c<p_c}
	\end{equation}
	This compensator has a zero at $z=z_c$ and a pole at $z=p_c$. The gain factor $(1-p_c)/(1-z_c)$ is included so that the low frequency or steady state gain $P_\mathrm{Lag}(1)=1$, analogous to the continuous-time lag compensator.

	Digital lag and lead compensators can be designed directly from $s$-domain specifications by using the transform between the $s$-domain and the $z$-domain defined by $z=e^{sT}$.
\end{example}


\begin{example}
	A proportional (P) controller has an output $u$ proportional to its input $e$, that is, $u=K_pe$, where $K_p$ is the proportionality constant. A derivative (D) controller has an output proportional to the derivative of its input $e$, that is, $u=K_Dde/dt$, where $K_D$ is a proportionality constant. An integral (I) controller has an output $u$ proportional to the integral of its input $e$, that is, $u=K_I\int{e(t)\,dt}$, where $K_I$ is a proportionality constant.  PD, PI, DI, and PID controllers are combinations of proportional (P), derivate (D), and integral (I) controllers. For example, the output $u$ of a PD controller has the form
	\begin{equation}
		u_{PD}=K_Pe+K_D\frac{de}{dt}
	\end{equation}
and the output of a PID controller has the form
	\begin{equation}
		u_{PD}=K_Pe+K_D\frac{de}{dt}+K_I\int{e(t)\,dt}
	\end{equation}
	The transfer function of this PID controller is
	\begin{equation}
		P_\mathrm{PID}(s)\equiv\frac{U_\mathrm{PID}(s)}{E(s)}=K_P+K_Ds+\frac{K_I}{s}=\frac{K_Ds^2+K_Ps+K_I}{s}
	\end{equation}
	This controller has two zeros and one pole. It is similar to the lag-lead compensator of the previous example except that the smallest pole is at the origin (an integrator) and it does nto have the second pole. It is typically mechanized in an analog or digital computer. 
\end{example}

\section{Root Locus Analysis and Design}

\section{Bode Analysis and Design}

\section{Miscellaneous Topics}

\subsection{Non-linear Control Systems}

\subsection{Controllability and Observability}

\subsection{State Feedback}

\subsection{Random Inputs}

\subsection{Optimal Control Systems}

\subsection{Adaptive Control Systems}

\chapter{Image Processing}

\section{Image Registration}

Brown divides image registration into four classes of problems: (1) multimodal registration, (2) template matching, (3) viewpoint registration and (4) temporal registration. Multimodal registration is registration of the same scene acquired from different sensors. Template registration is finding a match for a reference pattern in an image. Viewpoint registration is registration of images taken from different viewpoints. Finally, temporal registration is registration of the same scene taken at different times or under different conditions.

Given two images $I_1(x,y)$ and $I_2(x,y)$, image registration is the mapping between the two images which can be expressed as
\begin{align}
	I_2(x,y)&=g(I_1(f(x,y)))\\
	&=g(I_1(f_x(x,y),f_y(x,y)))
\end{align}

\backmatter

\end{document}

