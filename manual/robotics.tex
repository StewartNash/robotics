\usepackage{amsmath}
\usepackage{amssymb}

\documentclass[11pt]{book}

\begin{document}

\title{Robotics}
\author{Stewart Nash}

\maketitle

\tableofcontents

\mainmatter

\chapter{Dynamics}

\chapter{Controls}

\section{State Variable Representation}

Powerful tools from matrix algebra can be used to solve sets of first-order differential equations. That is why it is sometimes helpful to transform a system described by \emph{n}th-order differential equations into a system of first-order differential equations.

\subsection{Systems Modeled by Linear Differential Equations}

Consider a system represented by a \emph{n}th-order, single-input linear constant coefficient differential equation
\begin{equation}
	\sum{i=0}^n{a_i\frac{d^iy}{dt^i}}=u
\end{equation}
This equation can be replaced by \emph{n} first-order differential equations
\begin{equation}
	\left\{
		\begin{aligned}
			\frac{dx_k}{dt}&=x_{k+1},\,1<k<n\\
			\frac{dx_n}{dt}&==\frac{1}{a_n}\left[\sum_{k=0}^{n-1}{a_kx_{k+1}}\right]+\frac{1}{a_n}u
		\end{aligned}
	\right.
\end{equation}
where $x_1\def{y}$ and $i,k\in\mathbb{W}$. This can be written as a matrix equation
\begin{equation}
	\begin{bmatrix}
		\frac{dx_1}{dt}\\
		\frac{dx_2}{dt}\\
		\vdots\\
		\frac{dx_n}{dt}
	\end{bmatrix}
	\begin{bmatrix}
		0&1&0&\cdots&0\\
		0&0&1&\cdots&0\\
		\vdots&\vdots&\vdots&\ddots&\vdots\\
		-\frac{a_0}{a_n}&-\frac{a_1}{a_n}&-\frac{a_2}{a_n}&\cdots&-\frac{a_{n-1}}{a_n}
	\end{bmatrix}
	\begin{bmatrix}
		x_1\\
		x_2\\
		\vdots\\
		x_n
	\end{bmatrix}+
	\begin{bmatrix}
		0\\
		0\\
		\vdots\\
		0\\
		\frac{1}{a_n}
	\end{bmatrix}u
\end{equation}
or
\begin{equation}
	\frac{d\mathbf{x}}{dt}=A\mathbf{x}+\mathbf{b}u
\end{equation}
A multi-input-multi-output (MIMO) system can be represented by
\begin{equation}
	\begin{bmatrix}
		\frac{dx_1}{dt}\\
		\frac{dx_2}{dt}\\
		\vdots\\
		\frac{dx_n}{dt}
	\end{bmatrix}
	\begin{bmatrix}
		a_{11}&a_{12}&\cdots&a_{1n}\\
		a_{21}&a_{22}&\cdots&a_{2n}\\
		\vdots&\vdots&\ddots&\vdots\\
		a_{n1}&a_{n2}&\cdots&a_{nn}
	\end{bmatrix}
	\begin{bmatrix}
		x_1\\
		x_2\\
		\vdots\\
		x_n
	\end{bmatrix}+
	\begin{bmatrix}
		b_{11}&b_{12}&\cdots&b_{1r}\\
		b_{21}&b_{22}&\cdots&b_{2r}\\
		\vdots&\vdots&\ddots&\vdots\\
		b_{n1}&b_{n2}&\cdots&b_{nr}
	\end{bmatrix}
	\begin{bmatrix}
		u_1\\
		u_2\\
		\vdots\\
		u_r
	\end{bmatrix}
\end{equation}
or
\begin{equation}
	\frac{d\mathbf{x}}{dt}=A\mathbf{x}+B\mathbf{u}
\end{equation}
where $\mathbf{u}$ is an \emph{r}-vector of input functions.
Let $\mathbf{\Phi}$ be the $n\times{n}$ \emph{transition matrix} of the differential equation given above which is described by the matrix equation
\begin{equation}
	\frac{d\mathbf{\Phi}}{dt}=A\mathbf{\Phi}
\end{equation}

\subsection{Systems Modeled by Constant Coefficient Linear Difference Equations}

\section{Signal Flow Graphs}

\section{Nyquist Analysis and Design}

\section{Root Locus Analysis and Design}

\section{Bode Analysis and Design}

\chapter{Image Processing}

\backmatter

\end{document}

