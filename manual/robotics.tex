\documentclass[11pt]{book}

\usepackage{amsmath}
\usepackage{amssymb}
\usepackage{mathtools}

\begin{document}

\title{Robotics}
\author{Stewart Nash}

\maketitle

\tableofcontents

\mainmatter

\chapter{Dynamics}

\section{Kinematics}
Take a vector with components $\mathbf{P}=(a_x,b_y,c_z)$. A scale factor $w$ can be added (to the matrix form) to give
\begin{equation}
	\mathbf{P}=
	\begin{bmatrix}
		P_x\\
		P_y\\
		P_z\\
		w
	\end{bmatrix}
\end{equation}
where $(a_x,b_y,c_z)=(P_x/w,P_y/w,P_z/w)$. A direction vector can be represented by a scale factor of zero ($w=0$).

A universe reference frame is represented by $F_{x,y,z}$ and a moving frame is represented by $F_{n,o,a}$ where the letters n, o, and a come from the words normal, orientation and approach. Relative to the gripper, the $z$-axis is the approach axis by which the gripper approaches an object. The orientation with which the gripper frame approaches the part is the orientation axis. The normal-axis or x-axis is normal to both. A fourth vector which gives the location of a frame relative to a reference frame can be added to the vectors representing the components of the $n$-, $o$-, and $a$-axes to give a homogeneous matrix representation of this relative frame
\begin{equation}
	F=
	\begin{bmatrix}
		1&0&0&d_x\\
		0&1&0&d_y\\
		0&0&1&d_z\\
		0&0&0&1
	\end{bmatrix}
\end{equation}
Pre-multiplying the frame matrix by the transformation matrix will yield the new location of the frame.

Rotation matrices about the $x$-, $y$- and $z$-axes are given by
\begin{equation}
	\begin{aligned}
		\mathrm{rot}(x,\theta)&=
		\begin{bmatrix}
			1&0&0\\
			0&\cos{\theta}&\sin{\theta}\\
			0&\sin{\theta}&\cos{\theta}
		\end{bmatrix}\\
		\mathrm{rot}(y,\theta)&=
		\begin{bmatrix}
			\cos{\theta}&0&\sin{\theta}\\
			0&1&0\\
			-\sin{\theta}&0&\cos{\theta}
		\end{bmatrix}\\
		\mathrm{rot}(z,\theta)&=
		\begin{bmatrix}
			\cos{\theta}&-\sin{\theta}&0\\
			\sin{\theta}&\cos{\theta}&0\\
			0&0&1
		\end{bmatrix}
	\end{aligned}
\end{equation}
Denoting the transformation of frame $R$ relative to frame $U$ (universe) as $\prescript{U}{}T_R$, denoting the $p$ relative to the frame $R$ as $\prescript{R}{}p=p_{noa}$, and denoting $p$ relative to frame $U$ as $\prescript{U}{}p=p_{xyz}$ we have
\begin{equation}
	\prescript{U}{}p=\prescript{U}{}T_R\times\prescript{R}{}p
\end{equation}

\section{Differential Motion}

\section{Dynamic Analysis}

The Lagrangian is given by $L=K-P$ where $K$ and $P$ are the kinetic and potential energy of a system, respectively. If $F_i$ is the summation of all external forces for a linear motion and $T_i$ is the summation of all external forces for a linear motion and $T_i$ is the summation of all external torques for a rotational motion, then
\begin{equation}
	\begin{aligned}
		F_i&=\frac{\partial}{\partial{t}}\left(\frac{\partial{L}}{\partial\dot{x}_i}\right)-\frac{\partial{L}}{\partial{x_i}}\\
		T_i&=\frac{\partial}{\partial{t}}\left(\frac{\partial{L}}{\partial\dot{\theta}_i}\right)-\frac{\partial{L}}{\partial\theta_i}
	\end{aligned}
\end{equation}
The equation of motion for a 2-DOF system is given by
\begin{equation}
	\begin{bmatrix}
		T_i\\
		T_j
	\end{bmatrix}=
	\begin{bmatrix}
		D_{ii}&D_{ij}\\
		D_{ji}&D_{jj}
	\end{bmatrix}
	\begin{bmatrix}
		\ddot{\theta}_i\\
		\ddot{\theta}_j
	\end{bmatrix}+
	\begin{bmatrix}
		D_{iii}&D_{ijj}\\
		D_{jii}&D_{jjj}
	\end{bmatrix}
	\begin{bmatrix}
		\dot{\theta}_i\\
		\dot{\theta}_j
	\end{bmatrix}+
	\begin{bmatrix}
		D_{iij}&D_{iji}\\
		D_{jij}&D_{jji}
	\end{bmatrix}
	\begin{bmatrix}
		\dot{\theta}_i\dot{\theta}_j\\
		\dot{\theta}_j\dot{\theta}_i
	\end{bmatrix}+
	\begin{bmatrix}
		D_i\\
		D_j
	\end{bmatrix}
\end{equation}
Where the coefficient $D_{ii}$ is the effective inertia at joint $i$, such that an acceleration at joint $i$ causes a torque at joint $i$ equal to $D_{ii}\ddot{\theta}_i$, and the coefficient $D_{ij}$ is the coupling inertia between joints $i$ and $j$ such that an acceleration at joint $i$ or $j$ causes a torque at joint $j$ or $i$ equal to $D_{ij}\ddot{\theta}_j$ or $D_{ji}\ddot{\theta}_i$. $D_{ijj}\dot{\theta}_j^2$ terms represent centripetal forces acting at joint $i$ due to a velocity at joint $j$. All terms with $\dot{\theta}_i\dot{\theta}_j$ represent Coriolis accelerations and, when multiplied by corresponding inertias, represent Coriolis forces. $D_i$ represents gravity forces at joint $i$.

\section{Trajectory Planning}

\chapter{Controls}

\section{State Variable Representation}

Powerful tools from matrix algebra can be used to solve sets of first-order differential equations. That is why it is sometimes helpful to transform a system described by \emph{n}th-order differential equations into a system of first-order differential equations.

\subsection{Systems Modeled by Linear Differential Equations}

Consider a system represented by a \emph{n}th-order, single-input linear constant coefficient differential equation
\begin{equation}
	\sum_{i=0}^n{a_i\frac{d^iy}{dt^i}}=u
\end{equation}
This equation can be replaced by \emph{n} first-order differential equations
\begin{equation}
	\left\{
		\begin{aligned}
			\frac{dx_k}{dt}&=x_{k+1},\,1<k<n\\
			\frac{dx_n}{dt}&=\frac{1}{a_n}\left[\sum_{k=0}^{n-1}{a_kx_{k+1}}\right]+\frac{1}{a_n}u
		\end{aligned}
	\right.
\end{equation}
where $x_1\equiv{y}$ and $i,k\in\mathbb{W}$. This can be written as a matrix equation
\begin{equation}
	\begin{bmatrix}
		\frac{dx_1}{dt}\\
		\frac{dx_2}{dt}\\
		\vdots\\
		\frac{dx_n}{dt}
	\end{bmatrix}
	\begin{bmatrix}
		0&1&0&\cdots&0\\
		0&0&1&\cdots&0\\
		\vdots&\vdots&\vdots&\ddots&\vdots\\
		-\frac{a_0}{a_n}&-\frac{a_1}{a_n}&-\frac{a_2}{a_n}&\cdots&-\frac{a_{n-1}}{a_n}
	\end{bmatrix}
	\begin{bmatrix}
		x_1\\
		x_2\\
		\vdots\\
		x_n
	\end{bmatrix}+
	\begin{bmatrix}
		0\\
		0\\
		\vdots\\
		0\\
		\frac{1}{a_n}
	\end{bmatrix}u
\end{equation}
or
\begin{equation}
	\frac{d\mathbf{x}}{dt}=A\mathbf{x}+\mathbf{b}u
\end{equation}
A multi-input-multi-output (MIMO) system can be represented by
\begin{equation}
	\begin{bmatrix}
		\frac{dx_1}{dt}\\
		\frac{dx_2}{dt}\\
		\vdots\\
		\frac{dx_n}{dt}
	\end{bmatrix}
	\begin{bmatrix}
		a_{11}&a_{12}&\cdots&a_{1n}\\
		a_{21}&a_{22}&\cdots&a_{2n}\\
		\vdots&\vdots&\ddots&\vdots\\
		a_{n1}&a_{n2}&\cdots&a_{nn}
	\end{bmatrix}
	\begin{bmatrix}
		x_1\\
		x_2\\
		\vdots\\
		x_n
	\end{bmatrix}+
	\begin{bmatrix}
		b_{11}&b_{12}&\cdots&b_{1r}\\
		b_{21}&b_{22}&\cdots&b_{2r}\\
		\vdots&\vdots&\ddots&\vdots\\
		b_{n1}&b_{n2}&\cdots&b_{nr}
	\end{bmatrix}
	\begin{bmatrix}
		u_1\\
		u_2\\
		\vdots\\
		u_r
	\end{bmatrix}
\end{equation}
or
\begin{equation}
	\frac{d\mathbf{x}}{dt}=A\mathbf{x}+B\mathbf{u}\label{eq:3_25}
\end{equation}
where $\mathbf{u}$ is an \emph{r}-vector of input functions.\\
Let $\mathbf{\Phi}$ be the $n\times{n}$ \emph{transition matrix} of the differential equation given above which is described by the matrix equation
\begin{equation}
	\frac{d\mathbf{\Phi}}{dt}=A\mathbf{\Phi}
\end{equation}
If $\mathbf{\Phi}(0)=I$ (initial condition) then $\mathbf{\Phi}(t)=e^{At}$ where
\begin{equation}
	e^{At}=\sum_{n=0}^\infty{\frac{A^nt^n}{n!}}
\end{equation}
The solution to \eqref{eq:3_25} on the interval $0\leq{t}<\infty$ is given by
\begin{equation}
	\mathbf{x}(t)=e^{At}\mathbf{x}(0)+\int_0^t{e^{A(t-\tau)}B\mathbf{u}(\tau)\,d\tau}
\end{equation}

\subsection{Systems Modeled by Constant Coefficient Linear Difference Equations}

An $n$-th order (linear constant-coefficient) difference equation is given by
\begin{equation}
	\sum_{i=0}^n{a_iy(k+i)}=\sum_{i=0}^m{b_iu(k+i)}
\end{equation}
Define a shift operator by the equation
\begin{equation}
	Z[y(k)]\equiv{y(k+1)}
\end{equation}
The $n$-th order linear constant-coefficient difference equation
\begin{equation}
	y(k+n)+\sum_{i=0}^{n-1}{a_iy(k+i)}=u(k)
\end{equation}
can be written as
\begin{equation}
	(Z^n+\sum_{i=0}^{n-1}{a_iZ^i})[y(k)]=u(k)
\end{equation}
The characteristic equation of this difference equation is
\begin{equation}
	Z^n+\sum_{i=0}^{n-1}{a_iZ^i}=0
\end{equation}

\section{Signal Flow Graphs}

\section{Nyquist Analysis and Design}

\subsection{Mapping}

Let us consider a complex variable $s=\sigma+j\omega$. We will denote a complex transfer function of $s$ as $P(s)$. Let us also consider a complex variable $z=\mu+j\nu$ and denote a discrete-time (system) complex transfer function of $z$  as $P(z)$. For the first variable and transfer function we create two graphs: (1) the $s$-plane which has $j\omega$ on the ordinate and $\sigma$ on the abscissa, (2) the $P(s)$-plane which has $\mathrm{Im}\,P$ on the ordinate and $\mathrm{Re}\,P$ on the abscissa. The function $P$ maps points of the $s$-plane into the $P(s)$-plane. Similarly, $P(z)$ is a mapping or transformation from the $z$-plane to the $P(z)$-plane. For Nyquist stability plots, the locus of points in the $s$-plane which are chosen to map is called the Nyquist path. A polar plot is constructed in the $P(s)$-plane by taking $s=0+j\omega$.  

\section{Root Locus Analysis and Design}

\section{Bode Analysis and Design}

\section{Miscellaneous Topics}

\subsection{Non-linear Control Systems}

\subsection{Controllability and Observability}

\subsection{State Feedback}

\subsection{Random Inputs}

\subsection{Optimal Control Systems}

\subsection{Adaptive Control Systems}

\chapter{Image Processing}

\backmatter

\end{document}

